\documentclass{beamer}
\beamertemplatenavigationsymbolsempty
\usecolortheme{beaver}
\setbeamertemplate{blocks}[rounded=true, shadow=true]
\setbeamertemplate{footline}[page number]
%
\usepackage[utf8]{inputenc}
\usepackage[english, russian]{babel}
\usepackage{amssymb,amsfonts,amsmath,mathtext}
\usepackage[]{algorithmic}
\usepackage{subfig}
\usepackage[all]{xy} % xy package for diagrams
\usepackage{array}
\usepackage{tikz}
\usepackage{multicol}% many columns in slide
\usepackage{hyperref}% urls
\usepackage{hhline}%tables
% Your figures are here:
\graphicspath{ {fig/} {../fig/} }

%----------------------------------------------------------------------------------------------------------
\title{Распознавание галлюцинаций языковых моделей}
\author[К.\,Е. Петрушина]{Ксения Евгеньевна Петрушина}
\institute{Московский физико-технический институт,\\ Сколковский институт науки и технологий}

\date{\footnotesize

\par\smallskip\emph{Научный руководитель:} А.\,И.~Панченко
\par\bigskip\small 2023}
%----------------------------------------------------------------------------------------------------------
\begin{document}
%----------------------------------------------------------------------------------------------------------
\begin{frame}
\thispagestyle{empty}
\maketitle
\end{frame}
%-----------------------------------------------------------------------------------------------------
\begin{frame}{Цель исследования}
\begin{block}{Цель}
  Распознавание галлюцинаций языковых моделей в задачах машинного перевода, перефразирования и определения значения слова.
\end{block}
\begin{block}{Задача}
  Исследовать методы распознавания галлюцинаций в сгенерированном тексте и провести сравнительный анализ методов.
\end{block}
\end{frame}
%-----------------------------------------------------------------------------------------------------

\begin{frame}{Постановка задачи}

Дана выборка
\[\mathcal{D} = \{ (\text{src}_i, \text{hyp}_i, \text{label}_i) \}_{i=1}^N,\]

где $\text{src}_i$ -- входные данные языковой модели, $\text{hyp}_i$ -- ответ языковой модели, $\text{label}_i \in \{0, 1\}$ -- целевая переменная, обозначающая является ли ответ модели галлюцинацией.\\

Необходимо построить модель классификации \[f(\text{src}_i, \text{hyp}_i) = p_i \in [0, 1]\]

Метрикой качества является \[\text{Accuracy} =\dfrac1N \sum\limits_{i=1}^N (\mathbb{I} [f(\text{src}_i, \text{hyp}_i) \ge thr] = \text{label}_i)\]


\end{frame}
%----------------------------------------------------------------------------------------------------------

\begin{frame}{Список литературы}
\begin{enumerate}
    \item Patrick Lewis et al. 2020. \textcolor{blue}{Retrieval-augmented generation for knowledge-intensive NLP tasks}. % In Proceedings of the 34th International Conference on Neural Information Processing Systems (NIPS'20). Curran Associates Inc., Red Hook, NY, USA, Article 793, 9459–9474.
    
    URL: \url{https://dl.acm.org/doi/pdf/10.5555/3495724.3496517}

    \item David Dale et al. 2023. \textcolor{blue}{HalOmi: A Manually Annotated Benchmark for Multilingual Hallucination and Omission Detection in Machine Translation}.
    
    URL: \url{https://arxiv.org/pdf/2305.11746.pdf}

    %\item Nikolay Babakov, David Dale, Varvara Logacheva, and Alexander Panchenko. 2022. \textcolor{blue}{A large-scale computational study of content preservation measures for text style transfer and paraphrase generation}. %In Proceedings of the 60th Annual Meeting of the Association for Computational Linguistics: Student Research Workshop, pages 300–321, Dublin, Ireland. Association for Computational Linguistics.

    \item Albert Q. Jiang et al. 2023. \textcolor{blue}{Mistral 7B}.
    
    URL: \url{https://arxiv.org/pdf/2310.06825.pdf}

  \end{enumerate}
\end{frame}
%----------------------------------------------------------------------------------------------------------
\begin{frame}{Результаты экспериментов}

\begin{table}[]
    \centering
    \begin{tabular}{c|c}
    Method & Accuracy \\
    \hline
    Sentence transformer & \textbf{0.696}\\
    BERTScore f1 & 0.656\\
    Mistral-7B & 0.640\\
    \end{tabular}
    \caption{Paraphrase generation task}
    \label{tab:pg}
\end{table}

\begin{table}[]
    \centering
    \begin{tabular}{c|c}
    Method & Accuracy \\
    \hline
    LaBSE-en-ru & 0.786 \\
    BLASER 2.0-QE & \textbf{0.802} \\
    Mistral-7B & 0.684\\
    \end{tabular}
    \caption{Machine translation task}
    \label{tab:mt}
\end{table}

\begin{table}[]
    \centering
    \begin{tabular}{c|c}
    Method & Accuracy \\
    \hline
    Token-RAG + BERTScore f1 & 0.598 \\
    Wiktionary definition + E5 & \textbf{0.654} \\
    \end{tabular}
    \caption{Definition modeling task}
    \label{tab:dm}
\end{table}

\end{frame}

%----------------------------------------------------------------------------------------------------------
\begin{frame}{Заключение}
  \begin{itemize}
      \item Были рассмотрены различные подходы к задаче распознавания галлюцинаций языковых моделей
      \item Основной фокус исследования направлен на измерение похожести предложений
  \end{itemize}
\end{frame}
%----------------------------------------------------------------------------------------------------------
\end{document} 
